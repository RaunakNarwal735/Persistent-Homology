%Editor: Raunak Narwal
\documentclass[12pt,a4paper]{article}

\usepackage[utf8]{inputenc}
\usepackage[T1]{fontenc}
\usepackage{lmodern}           
\usepackage{microtype}         
\usepackage[a4paper,margin=1in]{geometry} 
\usepackage{parskip}          
\usepackage{graphicx}
\usepackage{float}
\usepackage{booktabs}

\usepackage{titlesec}
\usepackage{fancyhdr}
\usepackage{caption}
\captionsetup{font=small,labelfont=bf}
\usepackage[hidelinks]{hyperref}
\usepackage{xcolor}
\usepackage[normalem]{ulem}

\renewcommand{\familydefault}{\rmdefault}
\usepackage[T1]{fontenc}
\usepackage{lmodern}
\usepackage{tcolorbox} 
\usepackage{titling}   
\newtcolorbox{titlebox}{
  colback=orange!10,
  colframe=orange!80,
  boxrule=0.95pt,
  arc=6pt,
  auto outer arc,
  boxsep=6pt,
  left=10pt,
  right=10pt,
  top=10pt,
  bottom=10pt,
  colbacktitle=white,
  coltitle=black,
  center title
}


% Redefine maketitle
\pretitle{\begin{center}\begin{titlebox}\LARGE\bfseries}
\posttitle{\end{titlebox}\end{center}}
\preauthor{\begin{center}\large}
\postauthor{\end{center}}
\predate{\begin{center}}
\postdate{\end{center}}

\title{\textbf{}\\[0.5em]
}
\author{Raunak Narwal\\
Department of Mathematical Sciences \\
Indian Institute of Science Education and Research, Mohali, 130406, Punjab}

\date{ December 12}

% Section spacing and look
\titleformat{\section}{\large\bfseries}{\thesection}{1em}{}
\titleformat{\subsection}{\normalsize\bfseries}{\thesubsection}{1em}{}
\titlespacing*{\section}{0pt}{1.2\baselineskip}{0.4\baselineskip}
\titlespacing*{\subsection}{0pt}{0.9\baselineskip}{0.25\baselineskip}

% Header / footer (minimal)
\pagestyle{fancy}
\fancyhf{}
\renewcommand{\headrulewidth}{0pt}
\fancyhead[L]{\small Biweekly Report V}
\fancyhead[R]{\small Raunak Narwal}
\fancyfoot[C]{\small Page \thepage}\setlength{\footskip}{20pt}


% Utility for graphics with special filenames
\newcommand{\imgfile}[1]{\detokenize{#1}}

% ----- DOCUMENT -----
\begin{document}
\maketitle
\vspace{-0.8em}
\hrule
\vspace{1.0em}
%
\setlength{\parskip}{0.6\baselineskip}

\section*{Introduction}
We have added comments, docstrings and created UML diagrams for all of our scripts. UML files were created using \href{https://www.planttext.com/}{planttext} website, it take PUML files and outputs PNGs and SVGs.
I have added Numpy styled Docstrings using Pyment library, afterwards i have manually edited them to make them more descriptive and useful. I would try to include all the basics that build up to make our code. The information in the scripts, Readme file and this report overlaps.
Each of them has their own purpose and is useful in its own way.
\section*{Validation on Small Networks}
To be continued, it does create lots of errors such as SBML (XML) files dont have a KEGG ID so they have zero species in common which makes dynamic comparison invalid, for topological comparison, we don't yet have a framework to convert SBML files into CSV, so all of these would be done in the next report along with other pending agendas.
\section*{ Future Work Timeline}
 
\begin{table}[H]
    \centering
    \begin{tabular}{ccc}
        Event & Timeline \\
        Biweekly Report 1 & 21 August  \\
        Biweekly Report 1.1 &  25 August \\
        Mid semester exams &  1-2 September \\
        Biweekly Report 2 & \sout{3 September} 7 September  \\
        Mid semester exams &  \sout{3 September}  10 September  \\
        Awaiting Feedback &   September-October \\
        Mid semester II exams &    10-16 October \\
        Biweekly Report 3 &  \sout{21 September} 24 October  \\
        Biweekly Report 4 & 7 November \\
        End Semester Exams & 14-26 November \\
        Biweekly Report 4.1 & 2 December \\
    \end{tabular}
    \label{tab:placeholder}
\end{table}
\vfill
\noindent{\small Generated: \today}

\end{document}
