
\documentclass[12pt,a4paper]{article}

\usepackage[utf8]{inputenc}
\usepackage[T1]{fontenc}
\usepackage{lmodern}            % clean, well-proportioned fonts (pdflatex-safe)
\usepackage{microtype}         % better justification
\usepackage[a4paper,margin=1in]{geometry} 
\usepackage{parskip}           % space between paragraphs, no indent
\usepackage{graphicx}
\usepackage{float}
\usepackage{booktabs}

\usepackage{titlesec}
\usepackage{fancyhdr}
\usepackage{caption}
\captionsetup{font=small,labelfont=bf}
\usepackage[hidelinks]{hyperref}
\usepackage{xcolor}
\usepackage[normalem]{ulem}
% ----- STYLING -----
\renewcommand{\familydefault}{\rmdefault} % serif body for a formal look
\usepackage[utf8]{inputenc}
\usepackage[T1]{fontenc}
\usepackage{lmodern}
\usepackage{tcolorbox} % For title box
\usepackage{titling}   % For custom title formatting
\newtcolorbox{titlebox}{
  colback=pink!25,
  colframe=pink!80,
  boxrule=0.95pt,
  arc=6pt,
  auto outer arc,
  boxsep=6pt,
  left=10pt,
  right=10pt,
  top=10pt,
  bottom=10pt,
  colbacktitle=white,
  coltitle=black,
  center title
}


% Redefine maketitle
\pretitle{\begin{center}\begin{titlebox}\LARGE\bfseries}
\posttitle{\end{titlebox}\end{center}}
\preauthor{\begin{center}\large}
\postauthor{\end{center}}
\predate{\begin{center}}
\postdate{\end{center}}

\title{\textit{Biweekly Report 4 }\\[0.5em]
Adding Gillespie stochastic simulations and validating on small networks}
\author{Raunak Narwal\\
Department of Mathematical Sciences \\
Indian Institute of Science Education and Research, Mohali, 130406, Punjab}

\date{ October 24 to November 7, 2025}

% Section spacing and look
\titleformat{\section}{\large\bfseries}{\thesection}{1em}{}
\titleformat{\subsection}{\normalsize\bfseries}{\thesubsection}{1em}{}
\titlespacing*{\section}{0pt}{1.2\baselineskip}{0.4\baselineskip}
\titlespacing*{\subsection}{0pt}{0.9\baselineskip}{0.25\baselineskip}

% Header / footer (minimal)
\pagestyle{fancy}
\fancyhf{}
\renewcommand{\headrulewidth}{0pt}
\fancyhead[L]{\small Biweekly Report IV}
\fancyhead[R]{\small Raunak Narwal}
\fancyfoot[C]{\small Page \thepage}\setlength{\footskip}{20pt}


% Utility for graphics with special filenames
\newcommand{\imgfile}[1]{\detokenize{#1}}

% ----- DOCUMENT -----
\begin{document}
\maketitle
\vspace{-0.8em}
\hrule
\vspace{1.0em}

% Recommended line spacing: slightly tighter than 1.5 but more airy than single
\setlength{\parskip}{0.6\baselineskip}

\section*{Gillespie Stochastic Simulation Algorithm (SSA)}
The Gillespie Stochastic Simulation Algorithm (SSA) is a computational method which is used to simulate the time evolution of chemical reacting systems, mainly when dealing with small populations of molecules where stochastic effects are significant. Unlike deterministic approaches that use  ODEs to model average behavior, Gillespie's SSA captures the inherent randomness of reaction events at the molecular level. \\
Our dynamic comparison framework was originally designed to parsea and compare deterministic reaction networks in KGML format using ODEs , it lacked a stochastic simulation component. To address this, we have integrated the Gillespie SSA into our existing framework, allowing us to simulate and compare the dynamics of biochemical networks under stochastic conditions. This enhancement enables us to capture the variability and noise present in real biological systems, providing a more comprehensive comparison between different reaction networks. 
In biochemical kinetics, deterministic ODEs model the average concentration dynamics of species over time:
\[
\frac{d[\mathbf{X}]}{dt} = f([\mathbf{X}], \mathbf{k})
\]
where $[\mathbf{X}]$ are concentrations and $\mathbf{k}$ are rate constants.

Gillespie SSA models the discrete molecular events that occur probabilistically:
\[
a_r = c_r \prod_i n_i^{\nu_{ri}}
\]
where $a_r$ is the propensity of reaction $r$, $c_r$ is the stochastic rate constant, $n_i$ are molecule counts, and $\nu_{ri}$ is the stoichiometry.

These two formulations are theoretically equivalent in the limit of large molecular populations and nicely mixed conditions. However, converting between them requires scaling by Avogadro's number $N_A$ and system volume $V$ which was a lot of conversions with our code.
The relationship between deterministic rate constants $k_r$ and stochastic rate constants $c_r$ is given by:
\[c_r = \frac{k_r}{N_A^{\sum_i \nu_{ri} - 1} V^{\sum_i \nu_{ri} - 1}}\]
where $\sum_i \nu_{ri}$ is the total number of reactant molecules in reaction $r$. This scaling ensures that the average behavior of the stochastic model matches the deterministic ODEs when molecule counts are large. \\
\textbf{Added Components in the Code} \\
SBML parser to support smaller toy networks for example repressilator, futile cycles, predator prey systems and more. This enables us to import any synthethic network directly from BioModels or similar repository. \\
Gillespie Stochastic Simulation engine that implements a new simulator, based on the standard SSA, but it is extended with MonteCarlo averaging (multiple independent runs), automatic volume estimation (that might be a disadvantage too), diagnostic and logs and optional animations. \\
Helper function of volume estimation was added to set a physcially consistent and stable system volume, it aimed to produce a target number of events within the window of simulation preventing frozen or explosive dynamics. New simulation controls are added in the CLI. --dynamic gillespie to run the SSA ,, --volume V for volume (if no volume is provided the code automatically cimputes it using the estimator, it is better to not provide volume to see comparisons between ODE and SSA), --animate yes for animations of the networks.\\
\textbf{Autoscaling Volume} \\
Our first code ran nicely and produced visually appealing results but the concentrations were way off and turns out we did not convert the molecules to concentrations, so we used fixed volumes for conversion for example $1e-15$, but this resulted in zero errors and flat lines of trajectory, clearly something was off. So we used defined an automatic volume estimator, the auto scaling approach dynamically determines a suitable volume to yield a manageable event count and biologically realistic concentrations.\\
\textbf{Deterministic Stochastic Convergence} \\
This relationship is a cornerstone of chemical kinetics theory:
\[
\lim_{N_A V \to \infty} \frac{n_i}{N_A V} = [X_i]
\]
At large $V$, the Gillespie average becomes identical to the ODE solution. \\

\textbf{Results} \\
We tested our Gillespie SSA and our ODE implementations on \textit{map00400} and \textit{map00900}, the results showed similarities in MSE and other error values, at higher mc runs the stochastic trajectories converged closer to the deterministic ones, confirming the theoretical expectation. \\

\begin{verbatim}
    "metrics for Gillespie SSA": {
    "dtw_avg": 6.198763329018459,
    "mse_avg": 2.1381405810718794,
    "corr_avg": 0.13299220130476633,
    }
    "metrics for ODE": {
    "dtw_avg": 6.204106038148442,
    "mse_avg": 2.1387896309983585,
    "corr_avg": 0.13306254705179113,
    }
    "metrics for SSA mc=10": {
    "dtw_avg": 6.1989763329018459,
    "mse_avg": 2.1384581210718794,
    "corr_avg": 0.13199220130476633,
    }

\end{verbatim}

\begin{figure}[H]
    \centering
    \begin{minipage}{0.49\linewidth}
        \centering
        \includegraphics[width=\linewidth]{\imgfile{20251107_233806/dynamics_net1.png}}
    \end{minipage}
    \hfill
    \begin{minipage}{0.49\linewidth}
        \centering
        \includegraphics[width=\linewidth]{\imgfile{20251023_141937/dynamics_net1.png}}
    \end{minipage}
    
\end{figure} \\
The left one is gillespie SSA and right one is ODE, both the dynamics are very similar, the fluctuations in SSA are due to stochasticity, but overall trends match well.
\begin{figure}[H]
    \centering
    \begin{minipage}{0.49\linewidth}
        \centering
        \includegraphics[width=\linewidth]{\imgfile{20251107_233806/dynamics_net2.png}}
    \end{minipage}
    \hfill
    \begin{minipage}{0.49\linewidth}
        \centering
        \includegraphics[width=\linewidth]{\imgfile{20251023_141937/dynamics_net2.png}}
    \end{minipage}
\end{figure}

\begin{figure}[H]
    \centering
    \begin{minipage}{0.49\linewidth}
        \centering
        \includegraphics[width=\linewidth]{\imgfile{20251107_233806/overlay.png}}
    \end{minipage}
    \hfill
    \begin{minipage}{0.49\linewidth}
        \centering
        \includegraphics[width=\linewidth]{\imgfile{20251023_141937/overlay.png}}
    \end{minipage}
\end{figure}
\begin{figure}[H]
    \centering
    \begin{minipage}{0.49\linewidth}
        \centering
        \includegraphics[width=\linewidth]{\imgfile{20251107_233806/similarity_dtw.png}}
    \end{minipage}
    \hfill
    \begin{minipage}{0.49\linewidth}
        \centering
        \includegraphics[width=\linewidth]{\imgfile{20251023_141937/similarity_dtw.png}}
    \end{minipage}
\end{figure}

\begin{figure}[H]
    \centering
    \begin{minipage}{0.49\linewidth}
        \centering
        \includegraphics[width=\linewidth]{\imgfile{20251107_233806/similarity_mse.png}}
    \end{minipage}
    \hfill
    \begin{minipage}{0.49\linewidth}
        \centering
        \includegraphics[width=\linewidth]{\imgfile{20251023_141937/similarity_mse.png}}
    \end{minipage}
\end{figure}
\begin{figure}[H]
    \centering
    \begin{minipage}{0.49\linewidth}
        \centering
        \includegraphics[width=\linewidth]{\imgfile{20251107_233806/similarity_corr.png}}
    \end{minipage}
    \hfill
    \begin{minipage}{0.49\linewidth}
        \centering
        \includegraphics[width=\linewidth]{\imgfile{20251023_141937/similarity_corr.png}}
    \end{minipage}
\end{figure}
The animation videos are also mostly similar.

\section*{Validation on Small Networks}
To be continued, it does create lots of errors such as SBML (XML) files dont have a KEGG ID so they have zero species in common which makes dynamic comparison invalid, for topological comparison, we don't yet have a framework to convert SBML files into CSV, so all of these would be done in the next report along with other pending agendas.
\section*{ Future Work Timeline}
 
\begin{table}[H]
    \centering
    \begin{tabular}{ccc}
        Event & Timeline \\
        Biweekly Report 1 & 21 August  \\
        Biweekly Report 1.1 &  25 August \\
        Mid semester exams &  1-2 September \\
        Biweekly Report 2 & \sout{3 September} 7 September  \\
        Mid semester exams &  \sout{3 September}  10 September  \\
        Awaiting Feedback &   September-October \\
        Mid semester II exams &    10-16 October \\
        Biweekly Report 3 &  \sout{21 September} 24 October  \\
        Biweekly Report 4 & 7 November \\
        End Semester Exams & 14-26 November \\
        Biweekly Report 4.1 & 2 December \\
    \end{tabular}
    \label{tab:placeholder}
\end{table}
\vfill
\noindent{\small Generated: \today}

\end{document}
